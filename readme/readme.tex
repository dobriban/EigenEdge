% !TEX TS-program = pdflatex
% !TEX encoding = UTF-8 Unicode

% This is a simple template for a LaTeX document using the "article" class.
% See "book", "report", "letter" for other types of document.

\documentclass[english,11pt]{article} % use larger type; default would be 10pt

\usepackage[utf8]{inputenc} % set input encoding (not needed with XeLaTeX)

%%% Examples of Article customizations
% These packages are optional, depending whether you want the features they provide.
% See the LaTeX Companion or other references for full information.

%%% PAGE DIMENSIONS
\usepackage{geometry} % to change the page dimensions
\geometry{letterpaper} % or letterpaper (US) or a5paper or....
\geometry{margin=0.75in} % for example, change the margins to 2 inches all round
% \geometry{landscape} % set up the page for landscape
%   read geometry.pdf for detailed page layout information

\usepackage{graphicx} % support the \includegraphics command and options

% \usepackage[parfill]{parskip} % Activate to begin paragraphs with an empty line rather than an indent

%%% PACKAGES
\usepackage{booktabs} % for much better looking tables
\usepackage{array} % for better arrays (eg matrices) in maths
\usepackage{verbatim} % adds environment for commenting out blocks of text & for better verbatim
%\usepackage{subfig} % make it possible to include more than one captioned figure/table in a single float
% These packages are all incorporated in the memoir class to one degree or another...
\usepackage{amsthm,amsmath, amssymb}
\usepackage[authoryear]{natbib}
\usepackage[colorlinks=true,citecolor=blue, urlcolor=blue,breaklinks]{hyperref}
\usepackage{babel}
\usepackage{caption}
\usepackage{subcaption}
\usepackage{graphicx} % support the \includegraphics command and options
\usepackage{color}
\usepackage{framed}
\usepackage{lscape}
\usepackage{rotating}
\usepackage{algorithm}
\usepackage[noend]{algpseudocode}

%%% HEADERS & FOOTERS
\usepackage{fancyhdr} % This should be set AFTER setting up the page geometry
\pagestyle{fancy} % options: empty , plain , fancy
\renewcommand{\headrulewidth}{0pt} % customise the layout...
\lhead{}\chead{}\rhead{}
\lfoot{}\cfoot{\thepage}\rfoot{}

\usepackage{amssymb,amsmath,amsfonts}
\usepackage{hyperref}
\usepackage{natbib}

\usepackage{chngcntr}
\counterwithout{figure}{section}
\counterwithout{table}{section}
%%% SECTION TITLE APPEARANCE
%\usepackage{sectsty}
%\allsectionsfont{\sffamily\mdseries\upshape} % (See the fntguide.pdf for font help)
% (This matches ConTeXt defaults)

%%% ToC (table of contents) APPEARANCE
%\usepackage[nottoc,notlof,notlot]{tocbibind} % Put the bibliography in the ToC
%\usepackage[titles,subfigure]{tocloft} % Alter the style of the Table of Contents
%\renewcommand{\cftsecfont}{\rmfamily\mdseries\upshape}
%\renewcommand{\cftsecpagefont}{\rmfamily\mdseries\upshape} % No bold!

%%% END Article customizations

%%% The "real" document content comes below...

\title{\textsc{EigenEdge} package for MATLAB}
\author{Edgar Dobriban\thanks{Department of Statistics, Stanford University, \texttt{dobriban@stanford.edu}} }

%\date{} % Activate to display a given date or no date (if empty),
         % otherwise the current date is printed 

\begin{document}
\maketitle
\tableofcontents
\section{Introduction}

The \textsc{EigenEdge} MATLAB package contains open source implementations methods for working with eigenvalue distributions of large random matrices. In particular, it contains the \textsc{Atomic} method to compute the limit empirical spectrum of sample covariance matrices \citep[proposed in][]{dobriban2015precise}. %These are methods for improving power in multiple testing via the use of prior information.  

\begin{itemize}
\item{Version: } 0.0.1
\item{Requirements: } Tested on MATLAB R2014a and R2014b. 
\item{Author: } Edgar Dobriban
\item{License: } \verb+GPL-3+
\end{itemize}

In addition, this package contains the code to reproduce all simulation results from the paper \cite{dobriban2015precise}. These are contained in the \verb+\Experiments+ folder.

\section{Installation}

Extract the archive in any folder, say to \verb+<path>+ . The main functions are in the \verb+Code+ directory, which needs to be on the Matlab path, along with all of its subfolders. This can be accomplished in at least three ways. First, you can add the following lines to your Matlab startup:

$$\verb+addpath('<path>/pvalue_weighting_matlab/Code')+$$
$$\verb+addpath('<path>/pvalue_weighting_matlab/Code/Basic')+$$
$$\verb+addpath('<path>/pvalue_weighting_matlab/Code/External Helper Code')+$$

The second option is to add those line to scripts that call functions in this package. The third option is to execute the  \verb+setpaths.m+ script every time you start a new session with the \textsc{EigenEdge} package. This will include the current folder in the Matlab search path.

An example computation is in the \verb+\Experiments\Examples\example.m+ file. This is described in Section \ref{example}.

This file is the main documentation for the package. To start, look at the example (Section \ref{example}) or at the methods implemented (Sections \ref{methods}).

\section{Example}
\label{example}

\section{P-value weighting methods}
\label{methods}

For each p-value weighting method, we assume we observe data $T_i \sim \mathcal{N}(\mu_i, 1)$ and test each null hypothesis $H_i: \mu_i \ge 0$ against $\mu_i <0$.  The p-value for testing $H_i$ is $P_i = \Phi(T_i)$, where $\Phi$ is the normal cumulative distribution function.  For a weight vector $w \in [0,\infty)^{J}$ and significance level $q \in [0,1]$, the weighted Bonferroni procedure rejects $H_{i}$ if $P_i \le q w_i$. Usual Bonferroni corresponds to $w_i=1$.

Each p-value weighting method assumes some additional independent information about $\mu_i$, and returns a weight vector $w$. These can then be used for weighted Bonferroni or other multiple testing procedures.

\subsection{Bayes}


Bayes p-value weights can be computed using:  \verb+[w, q_star, q_thresh, c]+ \verb+= bayes_weights(eta, sigma, q)+. The inputs specify the prior distribution of the means $\mu_i $ of the test statistics as:

$$\mu_i \sim \mathcal{N}(\eta_i,\sigma_i^2), \mbox{   } i \in \{1,\ldots, J\}$$ 

where:

\begin{itemize}
\item \verb+eta+:  a vector of length J, the estimated means of test statistics, derived from the prior data
\item \verb+sigma+:  a strictly positive vector of length J, the estimated standard errors of test statistics, derived from the prior data
\item \verb+q+: The weights are optimal if each hypothesis is tested at level $q$. For instance, if we want to control the FWER globally at $0.05$, then we should use $q = 0.05/J$.
\end{itemize}

The outputs are: 
\begin{itemize}
\item \verb+w+:  the optimal weights. A non-negative vector of length J.
\item \verb+q_star+: the value $q^*$ for which the weights are optimal. This may differ slightly from the original $q$ if $q$ is large.
\item \verb+q_thresh+: the largest value of $q$ for which the weights can be computed exactly 
\item \verb+c+:  the normalizing constant produced by solving the optimization problem.
\end{itemize}

This method was proposed in \cite{dobriban2015optimal}.

\bibliography{eigenvalues}
\bibliographystyle{apalike}

\end{document}
