% !TEX TS-program = pdflatex
% !TEX encoding = UTF-8 Unicode

% This is a simple template for a LaTeX document using the "article" class.
% See "book", "report", "letter" for other types of document.

\documentclass[english,11pt]{article} % use larger type; default would be 10pt

\usepackage[utf8]{inputenc} % set input encoding (not needed with XeLaTeX)

%%% Examples of Article customizations
% These packages are optional, depending whether you want the features they provide.
% See the LaTeX Companion or other references for full information.

%%% PAGE DIMENSIONS
\usepackage{geometry} % to change the page dimensions
\geometry{letterpaper} % or letterpaper (US) or a5paper or....
\geometry{margin=0.75in} % for example, change the margins to 2 inches all round
% \geometry{landscape} % set up the page for landscape
%   read geometry.pdf for detailed page layout information

\usepackage{graphicx} % support the \includegraphics command and options

% \usepackage[parfill]{parskip} % Activate to begin paragraphs with an empty line rather than an indent

%%% PACKAGES
\usepackage{booktabs} % for much better looking tables
\usepackage{array} % for better arrays (eg matrices) in maths
\usepackage{verbatim} % adds environment for commenting out blocks of text & for better verbatim
%\usepackage{subfig} % make it possible to include more than one captioned figure/table in a single float
% These packages are all incorporated in the memoir class to one degree or another...
\usepackage{amsthm,amsmath, amssymb}
\usepackage[authoryear]{natbib}
\usepackage[colorlinks=true,citecolor=blue, urlcolor=blue,breaklinks]{hyperref}
\usepackage{babel}
\usepackage{caption}
\usepackage{subcaption}
\usepackage{graphicx} % support the \includegraphics command and options
\usepackage{color}
\usepackage{framed}
\usepackage{lscape}
\usepackage{rotating}
\usepackage{algorithm}
\usepackage[noend]{algpseudocode}

%%% HEADERS & FOOTERS
\usepackage{fancyhdr} % This should be set AFTER setting up the page geometry
\pagestyle{fancy} % options: empty , plain , fancy
\renewcommand{\headrulewidth}{0pt} % customise the layout...
\lhead{}\chead{}\rhead{}
\lfoot{}\cfoot{\thepage}\rfoot{}

\usepackage{amssymb,amsmath,amsfonts}
\usepackage{hyperref}
\usepackage{natbib}

\usepackage{chngcntr}
\counterwithout{figure}{section}
\counterwithout{table}{section}
%%% SECTION TITLE APPEARANCE
%\usepackage{sectsty}
%\allsectionsfont{\sffamily\mdseries\upshape} % (See the fntguide.pdf for font help)
% (This matches ConTeXt defaults)

%%% ToC (table of contents) APPEARANCE
%\usepackage[nottoc,notlof,notlot]{tocbibind} % Put the bibliography in the ToC
%\usepackage[titles,subfigure]{tocloft} % Alter the style of the Table of Contents
%\renewcommand{\cftsecfont}{\rmfamily\mdseries\upshape}
%\renewcommand{\cftsecpagefont}{\rmfamily\mdseries\upshape} % No bold!

%%% END Article customizations

%%% The "real" document content comes below...

\title{Fractal Dynamics of Spectral Density Calculations}
\author{Edgar Dobriban\thanks{Department of Statistics, Stanford University, \texttt{dobriban@stanford.edu}} }

%\date{} % Activate to display a given date or no date (if empty),
         % otherwise the current date is printed 

\begin{document}
\maketitle
\tableofcontents
\section{Introduction}

This folder contains supplementary information on the complex dynamics
of methods for solving the Silverstein-Marchenko-Pastur equation.

We provide scripts to compute the Julia sets of iterative methods for 
this problem. These scripts are written in the language UltraFractal.

\begin{itemize}
\item{Version: } 0.0.1
\item{Requirements: } UltraFractal (tested on 5.04) 
\item{Author: } Edgar Dobriban
\item{License: } \verb+GPL-3+
\end{itemize}

\section{Contents}

There are two folders, corresponding to the Fixed Point and Newton Methods.

\subsection{ ./Fixed Point Method}
\begin{itemize}
\item \verb+./MP_Iter_Sol.ufm+ - This contains an UltraFractal formula. It programs the fixed
point iteration method for a population spectrum $H = \frac{1}{2}(\delta_1 + \delta_{10})$.
\item \verb+./MP_Iter_Julia_Set.png+ - This is a picture showing the Julia set of the dynamical
system. The Julia set (white) belongs to the negative complex half-lane $\{z: im(z)<0\}$, which
confirms empirically that the method converges for $z$ with positive imaginary part.

\begin{figure}
\centering
  \includegraphics[scale=1]{"../Fixed Point Method/MP_Iter_Julia_Set"}
\caption{Julia set of Fixed Point Method.}
\label{fig1}
\end{figure}

\end{itemize}

\subsection{ ./Newton Method}
\begin{itemize}
\item \verb+./MP_newton.ufm+ - This contains an UltraFractal formula. It programs Newton's 
method for a population spectrum $H =\delta_1$.
\item \verb+./high_resolution_MP_null.png+ - This is a picture showing the Julia set of the dynamical
system. The Julia set is now a complicated-looking fractal. This provides empirical
support for our claims in the accompanying paper that Newton's method is numerically sensitive.
\end{itemize}

\begin{figure}
\centering
  \includegraphics[scale=1]{"../Newton Method/high_resolution_MP_null"}
\caption{Julia set of Newton Method.}
\label{fig1}
\end{figure}

\end{document}
